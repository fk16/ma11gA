\documentclass[11pt,a4paper]{scrartcl}
\usepackage[T1]{fontenc}
\usepackage[utf8]{inputenc}
\usepackage{fancyhdr}
\usepackage{multicol}
\usepackage{ngerman}
\usepackage{setspace}\linespread{1.15}%\onehalfspacing%
\newlength{\zeile}\setlength{\zeile}{1em}
\setlength{\parindent}{0pt}\setlength{\parskip}{1em}
\newlength{\figline}
\setlength{\figline}{-1em}

%\usepackage{helvet}
\usepackage[default,scale=1]{opensans}
\usepackage{subfiles}
\usepackage{enumitem}
\usepackage[
	colorlinks=true, 
	linkcolor=black, % gray20, % Chart1, % 
	linktoc=all,
	urlcolor=black, %gray20, % Chart1, % 
	citecolor=black, %gray20 % Chart1 % 
	]{hyperref}

\usepackage{tikz,pgf}
\usepackage{scrextend}
\usepackage{longtable,multirow}
\usepackage{graphicx, wrapfig, subcaption}
\usepackage[font=small,labelfont=normalfont, figurename=Abb., tablename=Tab.]{caption} 
\usepackage{pdfpages, pdflscape} % enable pdf subfiles and pdf landscape pages

\usepackage{amsmath, amsfonts, amssymb}

\newenvironment{rcases}{\left.\begin{aligned}}{\end{aligned}\right\rbrace}
\newenvironment{lcases}{\left\lbrace\begin{aligned}}{\end{aligned}\right.}

\usepackage{makeidx}
\usepackage{cite}
%\usepackage[round]{natbib}
%\usepackage[style=numeric,citestyle=phys]{bibtex}

%\usepackage{biber}


\setuptoc{toc}{totoc}

\usepackage{todonotes}
\usepackage{pifont}

\usepackage{courier}
\usepackage{listings}
	\lstset{
		basicstyle=\ttfamily\footnotesize, 
		language=C++, 
		numbers=left,
		frame=single, 
		rulesepcolor=\color[gray]{.6},%\color[RGB]{0, 40, 80}, 
		breaklines=true, 
		keywordstyle=\bfseries\color[gray]{.25},%\color[RGB]{0, 60, 120},
		commentstyle=\color[gray]{.5},%\color[RGB]{30, 120, 30},
		morekeywords={Real, string},
		showspaces=false, 
		keepspaces=false,
	}

\usepackage[yyyymmdd]{datetime}
\renewcommand{\dateseparator}{-}
\usepackage[left=2.5cm,right=2.5cm, top=3cm, bottom=3cm]{geometry}

\begin{document}
\pagestyle{fancy}
\title{Übungen zur ersten Klausur 8c}
\author{Florian Knierim}

\lhead{\footnotesize
	Schuljahr 2019/20\\
	Klausur I: Lineare Gleichungen\\
	Termin: 17.10.2019
}

\rhead{\footnotesize
	Friedrich-Wilhelm-Schule\\
	\href{mailto:f.knierim@tutanota.com}{F. Knierim}\\
	Klasse 8c
}
\section*{Aufgaben}\vspace{-1em}
1. Benenne die ausgeführte Umformung und fülle die Dreiecke und Quadrate aus.
\begin{multicols}{4}
	\begin{enumerate}[label=\alph*)] 
		\item $x+12 = 38 $\\ 		$x = \square $
		\item $\frac{t}{7}-4 = 24$\\ $\frac{t}{7} = \triangle$\\ $t = \square$
		\item $k-1,7 = 9$\\ 		$k = \square $
		\item $3y -4 = 5$\\ $3y = \triangle $\\ $y = \square$
		\item $a\cdot 15 = 60 $\\	$a = \square $
		\item $z= 4y+17$\\$-3z=\triangle$\\$z = \square$
		\item $\frac{3x}{4} = 1$\\	$x= \square $
		\item $17+34d = 85d$\\ $\triangle = 51d$\\ $\square = d$
	\end{enumerate}
\end{multicols}

2. Überprüfe die Rechnungen, korrigiere gegebenenfalls.
\begin{multicols}{2}
	\begin{enumerate}[label=\alph*)] 
		\item $\frac{1}{4}a = \frac{4}{16} \;\;|\cdot 4\\ a = \frac{4\cdot4}{16}$
		\item $1,7x -4,5 = 29,5 \;\;|:1,7$\\ $x-4,5 = \frac{29,5}{1,7}$
		\item $ t = 5t -5 \;\;|+5$\\ $5t = 5t$
		\item $6x-17 = -2x+15\;\;|-2x$\\ $8x -17 = 15$
	\end{enumerate}
\end{multicols}
3. Im Buch auf Seite 20 die Aufgabe 18.

4. Löse die Gleichung und gib die Lösungsmenge an.
\begin{multicols}{2}
	\begin{enumerate}[label=\alph*)] 
		\item $5u+3-7u+17\cdot 3u - 34u+7 = 1$
		\item $0,25t+4-1,7t\cdot 2 -3,1 = -4,15t-4,3+t$
		\item $2+3x-4-5x-6=7x-8-9x$
		\item $-1+c+3-5c=7-9c+11+13c$
	\end{enumerate}
\end{multicols}

5. Löse die Klammern auf und vereinfache dann soweit wie es geht. (Bedenke, dass vor Klammern (wie vor Variablen) die Malpunkte weggelassen werden können. Das heißt es ist $4(x+1) = 4\cdot (x+1)$ und genauso $3x = 3\cdot x$.)
\begin{multicols}{3}
	\begin{enumerate}[label=\alph*)] 
		\item $9(x+1)+3(x+5)$ \item $-(x+1)+2$ \item $-4(\frac{1}{4}t-0,5)+3(t-1) $
		\item $x(4+5)-9(x+2)$ \item $\frac{1}{3}(c-6)+\frac{1}{9}(5+c)$ \item $(5c-3)\cdot 3 - 3 (4c-7)$
	\end{enumerate}
\end{multicols}

6. Löse die Gleichung nach der Variablen auf und benenne die Lösungsmenge.
\begin{multicols}{2}
	\begin{enumerate}[label=\alph*)] 
		\item $2(x+1)=4(x+2)-(x+1)$ \item $2-2(\frac{1}{4}t-0,75)=1,5(t-2)-6(3+t) $
		\item$x(4+5)=9(x+2)-\frac{1}{3}(x-6)+\frac{x}{3}$\item $3+\frac{1}{3}(5+c)=(5c-3)\cdot 3 - 3 (4c-7)$
	\end{enumerate}
\end{multicols}

\newpage 
\section*{Lösungen}
1.\vspace{-2em} %Benenne die ausgeführte Umformung und fülle die Dreiecke und Quadrate aus.
	\begin{enumerate}[noitemsep, label=\alph*)] 
		\item $|-12$ also $\square = 26$
		\item erst $|+4$ also $\triangle = 20$, dann $|\cdot7$ also $\square = 140$
		\item $|+1,7$ also $\square = 10,7$
		\item erst $|+4$ also $\triangle = 9$, dann $|:3$ also $\square = 3$ 
		\item $|:15$ also $\square = 4$
		\item erst $-4y$ also $\triangle=17$, dann $|:(-3)$ also $\square = \frac{-17}{3}$
		\item $|:\frac{3}{4}$ oder $|\cdot\frac{4}{3}$ also $\square = \frac{4}{3}$
		\item erst $|-34d$ also $\triangle = 17$, dann $|:51$ also $\square = \frac{1}{3}$
	\end{enumerate}

2.\vspace{-2em} %Überprüfe die Rechnungen, korrigiere gegebenenfalls.
	\begin{enumerate}[label=\alph*), noitemsep] 
		\item Die Rechnug ist grundsätzlich korrekt, es könnte aber noch weiter vereinfacht werden, weil $\frac{4\cdot4}{16}=\frac{16}{16}=1$ ist.
		\item Die Rechnung ist falsch, weil die Rechnung $|:1,7$ nur auf $1,7x$ und $29$ angewendet wurde aber nicht auf $-4,5$. \\ Besser wäre es zuerst die $|+4,5$ zu rechnen und so die $-4,5$ auf die rechte Seite zu bringen. Dann steht links nur etwas mit $x$ und rechts nur Zahlen. Erst dann sollte man den Vorfaktor vor $x$ eliminieren ($|:1,7$). Das Ergebnis wäre dann $x=20$.
		\item Die Rechnung ist falsch. Korrekt ausgeführt wäre das Ergebnis $t+5 = 5t$ und damit nicht zielführend.\\
		Besser wäre es $|-5t$ zu rechnen. Dann käme zunächst $-4t = -5$ und mit $|:(-4)$ das Ergebnis $t = 1,25$ heraus.  
		\item Die Rechnung ist falsch. Die korrekte Ansage wäre $|+2$ gewesen. Weiter ausgeführt ist das Ergebnis $x=4$.
	\end{enumerate}

3.\vspace{-2em} %Im Buch auf Seite 20 die Aufgabe 18.
\newcommand{\SOLN}[1]{$\mathbb{L} =\{#1\}$}
\begin{multicols}{4}
	\begin{enumerate}[noitemsep,label=\alph*)] 
		\item \SOLN{-2}
		\item \SOLN{14}
		\item \SOLN{21}
		\item \SOLN{-1}
		\item \SOLN{8}
		\item \SOLN{1}
		\item \SOLN{-0,5}
		\item \SOLN{0,9}
		\item \SOLN{20}
		\item \SOLN{\frac13}
		\item \SOLN{\frac13}
		\item \SOLN{4}
		\item \SOLN{\frac98}
		\item \SOLN{3}
		\item \SOLN{-1}
		\item \SOLN{-0,1}
		\item \SOLN{0,375}
		\item \SOLN{-2,25}
		\item \SOLN{22}
		\item \SOLN{1}
	\end{enumerate}
\end{multicols}

	

4.\vspace{-2em}% Löse die Gleichung und gib die Lösungsmenge an.
\begin{multicols}{2}
	\begin{enumerate}[label=\alph*), noitemsep] 
		\item \SOLN{-0,6}
		\item \SOLN{}, keine Zahl löst diese Gleichung.
		\item $\mathbb{L} = \mathbb{Q}$, jede rationale Zahl löst die Gln.
		\item \SOLN{-2}
	\end{enumerate}
\end{multicols}

5.\vspace{-2em} %Löse die Klammern auf und vereinfache dann soweit wie es geht. (Bedenke, dass vor Klammern, wie vor Variablen die Malpunkte weggelassen werden können. Das heißt es ist $4(x+1) = 4\cdot (x+1)$ und genauso $3x = 3\cdot x$.)
\begin{multicols}{3}
	\begin{enumerate}[label=\alph*), noitemsep] 
		\item $12x+24$ \item $-x+1$ \item $2t-1$
		\item $-18$ \item $\frac{4}{9}c-\frac{13}{9}$ \item $3c+12$
	\end{enumerate}
\end{multicols}

6.\vspace{-2em}% Löse die Gleichung nach der Variablen auf und benenne die Lösungsmenge.
\begin{multicols}{4}
	\begin{enumerate}[label=\alph*)] 
		\item \SOLN{-5} \item \SOLN{-6,125}
		\item \SOLN{} \item \SOLN{-2,75}
	\end{enumerate}
\end{multicols}
\end{document}